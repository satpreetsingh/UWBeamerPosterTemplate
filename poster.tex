% Gemini theme
% https://github.com/anishathalye/gemini
%
% We try to keep this Overleaf template in sync with the canonical source on
% GitHub, but it's recommended that you obtain the template directly from
% GitHub to ensure that you are using the latest version.

\documentclass[final]{beamer}

% ====================
% Packages
% ====================
\usepackage{lipsum}
\usepackage[T1]{fontenc}
\usepackage{lmodern}

% 48 inches in centimeters = 122
% 36 inches in centimeters = 92
% \usepackage[size=custom,width=122,height=92,scale=1.0]{beamerposter}

% A2 is 42 x 59.4
% \usepackage[size=custom,width=42,height=59.4,scale=1.0]{beamerposter}

% 2000x1300 px @ 72 dpi == 28 in x 18 in ::3X:: 56 x 36 
\usepackage[size=custom,width=84,height=54,scale=1.0]{beamerposter}

% \usetheme{mit}
% \usecolortheme{mit}
% \usetheme{gemini}
% \usecolortheme{gemini}
\usetheme{uw}
\usecolortheme{uw}


\usepackage{tabularx}
\usepackage{booktabs}
\usepackage{tikz}
\usetikzlibrary{positioning, shapes}
\usepackage{pgfplots}

% https://tex.stackexchange.com/questions/157389/how-to-center-column-values-in-a-table
\usepackage{array}
\newcolumntype{P}[1]{>{\centering\arraybackslash}p{#1}}


\usepackage{caption} % for caption*
% \usepackage[
% backend=biber,
% style=alphabetic,
% citestyle=authoryear
% ]{biblatex}

% \usepackage[style=nature,maxnames=1,uniquelist=false]{biblatex}


% ====================
% Lengths
% ====================

% If you have N columns, choose \sepwidth and \colwidth such that
% (N+1)*\sepwidth + N*\colwidth = \paperwidth
\newlength{\sepwidth}
\newlength{\colwidth}
\setlength{\sepwidth}{0.025\paperwidth}
\setlength{\colwidth}{0.30\paperwidth}

\newcommand{\separatorcolumn}{\begin{column}{\sepwidth}\end{column}}

% ====================
% Title
% ====================

\title{
Some very bright ideas
}



\author{Industrious Grad-Student \inst{1} 
% [satsingh@uw.edu]
\and 
Lucky Undergrad \inst{2} 
\and 
Supervising Advisor \inst{1}
}

\institute[shortinst]{
\inst{1} University of Washington, Seattle, WA 98195 
\samelineand 
\inst{2} Well Paying Co.
\quad Correspondence: \href{mailto:husky@uw.edu}{husky@uw.edu}
}

\begin{document}
\addtobeamertemplate{headline}{} 
{\begin{tikzpicture}[remember picture, overlay]
     % UW Logo on top left (West)
     \node [anchor=north west, inner sep=1.5cm, xshift=+3cm]  at (current page.north west)
     {\includegraphics[height=5.5cm]{Signature_Stacked_White.png}};
     
     % UNR Logo on top right (East)
     \node [anchor=north east, inner sep=2cm, xshift=-2cm]  at (current page.north east)
     {\includegraphics[height=5cm]{CC.png}};
\end{tikzpicture}}


\begin{frame}[t]
\begin{columns}[t]
\separatorcolumn

\begin{column}{\colwidth}

\begin{block}{Overview}
\lipsum[5]
\end{block}


% #### Question ####
\begin{center}
\setbeamercolor{postit}{fg=black,bg=lightgray}
\begin{beamercolorbox}[sep=1em, rounded=true, center]{postit}
\textbf{Question: 
\lipsum[2]
} 
\end{beamercolorbox}
\end{center}

\vspace{-0.5cm}
\begin{block}{First bright idea} 
\begin{figure}
    \centering
    \includegraphics[width=0.5\linewidth]{husky1.jpg}
    \captionsetup{justification=centering}
    \caption{ A very good boy }
    \label{husky1}
\end{figure}
\end{block}

\end{column}
\separatorcolumn

% ############ COLUMN 2 ############ 

\begin{column}{\colwidth}

\begin{block}{Surprising result} 
\begin{figure}
    \centering
    \includegraphics[width=0.95\linewidth]{husky2.jpg}
    \captionsetup{justification=centering}
    \caption{You did what using \LaTeX !?}
\label{fig_pomdp}
\end{figure}
\end{block}    

\vspace{-1.2cm}
\begin{block}{Second bright idea} 
\lipsum[8]
\end{block}


\end{column}
\separatorcolumn % Middle column margin

%%%%%%%%%%%%%%%%%%%%%%% COLUMN 3 %%%%%%%%%%%%%%%%%%%%%%% 
\begin{column}{\colwidth}

\begin{block}{Future plans}
\lipsum[5]
\end{block}

\begin{figure}
    \centering

\begin{tikzpicture}

	\node (1) [draw, minimum width=15em, minimum height=2em, very thick, rounded rectangle] {};
	\node (l1) [left=0em of 1] {$\vec{x}$};
		
	\node (2) [above=3.9em of 1, draw, fill=lightgray, minimum width=9em,very thick, minimum height=2em, rounded rectangle] {};
	\node (l2) [left=0em of 2] {$\vec{h}_1$};
	\node (3) [above=3.9em of 2, draw, fill=lightgray, minimum width=9em,very thick, minimum height=2em, rounded rectangle] {};
	\node (l3) [left=0em of 3] {$\vec{h}_2$};
		
	\node[circle, draw, thick] (A1) {};
	\node[circle, draw, thick, right=0.5em of A1] (A2) {};
	\node[circle, draw, thick, right=0.5em of A2] (A3) {};
	\node[circle, draw, thick, right=0.5em of A3] (A4) {};
	\node[circle, draw, thick, right=0.5em of A4] (A5) {};
	\node[circle, draw, thick, left=0.5em of A1] (A6) {};
	\node[circle, draw, thick, left=0.5em of A6] (A7) {};
	\node[circle, draw, thick, left=0.5em of A7] (A8) {};
	\node[circle, draw, thick, left=0.5em of A8] (A9) {};
		
	\node[circle, draw, fill=white, thick, above=5em of A1] (B1) {};
	\node[circle, draw, fill=white, thick, right=0.5em of B1] (B2) {};
	\node[circle, draw, fill=white, thick, right=0.5em of B2] (B3) {};
	\node[circle, draw, fill=white, thick, left=0.5em of B1] (B4) {};
	\node[circle, draw, fill=white, thick, left=0.5em of B4] (B5) {};
		
	\node[circle, draw, fill=white, thick, above=5em of A1] (B1) {};
	\node[circle, draw, fill=white, thick, right=0.5em of B1] (B2) {};
	\node[circle, draw, fill=white, thick, right=0.5em of B2] (B3) {};
	\node[circle, draw, fill=white, thick, left=0.5em of B1] (B4) {};
	\node[circle, draw, fill=white, thick, left=0.5em of B4] (B5) {};
		
	\node[circle, draw, fill=white, thick, above=5em of A1] (B1) {};
	\node[circle, draw, fill=white, thick, right=0.5em of B1] (B2) {};
	\node[circle, draw, fill=white, thick, right=0.5em of B2] (B3) {};
	\node[circle, draw, fill=white, thick, left=0.5em of B1] (B4) {};
	\node[circle, draw, fill=white, thick, left=0.5em of B4] (B5) {};
		
	\node[circle, draw, fill=white, thick, above=5em of B1] (C1) {};
	\node[circle, draw, fill=white, thick, right=0.5em of C1] (C2) {};
	\node[circle, draw, fill=white, thick, right=0.5em of C2] (C3) {};
	\node[circle, draw, fill=white, thick, left=0.5em of C1] (C4) {};
	\node[circle, draw, fill=white, thick, left=0.5em of C4] (C5) {};
		
	\foreach \x in {1,...,9}
		\foreach \y in {1,...,5}
			\draw[-stealth, thick] (A\x) -- (B\y);
				
	\foreach \x in {1,...,5}
		\foreach \y in {1,...,5}
			\draw[stealth-stealth, thick] (B\x) -- (C\y);
				
	\draw[-stealth, thick] (A5) -- node[right] {${\bf W}_1$} (B3);
	\draw[stealth-stealth, thick] (B3) -- node[right] {${\bf W}_2$} (C3);
	
\end{tikzpicture}

    \captionsetup{justification=centering}
    \caption{Buzzword}
\label{fig_pomdp}
\end{figure}



\vspace{+0.7cm}
\begin{block}{References}
\begin{tabular}{l l}
1. Author et al, 2020   & 4. Self citation, 2015  \\
2. Author et al, 2016   & 5. Self citation, 2018 \\
3. Author et al, 2018   & 6. Author et al, 2008\\
7. Another sneaky self-citation, 2006   &  \\
\end{tabular}
\end{block}

% \vspace{+1.0cm}
\begin{block}{Acknowledgements}
\small{
I modified the repository \url{https://github.com/anishathalye/gemini} to make this, and used stock images from Shutterstock.
}
\end{block}



\end{column}
\separatorcolumn % Right Margin

\end{columns}

\end{frame}

\end{document}
